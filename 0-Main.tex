\documentclass[Portugues]{projetoFAPESP}

\addbibresource{bibliografia.bib}

\setlength{\parindent}{4em}
\setlength{\parskip}{0.77em}

\usepackage[utf8]{inputenc}
\usepackage[english]{babel}
\usepackage{hyperref}
\usepackage{bookmark}
\usepackage{mathtools}
\usepackage{listings}
\usepackage{multirow}
\usepackage{graphicx}
\usepackage{float}
\usepackage{comment}
\usepackage{subcaption}
\usepackage{indentfirst}
\usepackage{caption}

\usepackage{colortbl}

%\usepackage[table,xcdraw]{xcolor}

% fonte Arial
% \usepackage{helvet}
% \renewcommand{\familydefault}{\sfdefault}

%% Página de título
\universidade{Universidade de São Paulo}
\faculdade{Escola de Engenharia de São Carlos}
\cidade{São Carlos}

\titulo{Posicionamento e exibição de imagens 3D utilizando óculos de realidade aumentada para aplicação cirúrgica}

\Proj{2020/15835-4}
\agFomento{Fundação de Amparo à Pesquisa do Estado de São Paulo}{FAPESP}
\modalidadeProjeto{de auxílio à iniciação científica}

\professor{Glauco Augusto de Paula Caurin}{Dr.}
\Coorientador{Paulo Henrique Polegato}{}
\beneficiario{Calvin Suzuki de Camargo}

\inicioPeriodoVigencia{01}{03}{2021}
\periodoRelatorio{28}{02}{2022}
\fimPeriodoVigencia{28}{02}{2022}

\begin{document}

\pagenumbering{roman}

\geraTitulo

\folhaDeRosto

\begin{resumo}

Trata-se do relatório final do projeto de pesquisa de iniciação científica que compreende os trabalhos iniciados em março de 2021 até o final de fevereiro de 2022. O projeto estuda a exibição de objetos 3D em óculos inteligentes (\textit{Smart glasses}) com realidade aumentada, que podem servir como um dispositivo auxiliar para aplicações cirúrgicas. Para isso, conceituamos as relações entre visão computacional e computação gráfica no campo da realidade aumentada. Com o apoio do Laboratório Aeronáutico de Tecnologias (AeroTech), definimos uma arquitetura de sistema que permite a sobreposição de um modelo 3D na cabeça de um paciente, indicando pontos de implantação de eletrodos durante o procedimento neurocirúrgico orientado por estereoeletroencefalografia (\textit{SEEG}). Essa arquitetura foi baseada na comunicação de computador e óculos pela rede local: O computador recebe as imagens da câmera dos óculos e utiliza o método de detecção por mercadores fiduciais (\textit{ArUco}) para estimar a posição do paciente e envia as coordenadas aos óculos, indicando onde a projeção deve ser exibida. Os resultados parciais da aplicação apresentam uma rápida responsividade com o método de estimação escolhido.

  \palavraschaves{\textit{Smart glasses}, Aplicação cirúrgica, Visão computacional, Realidade aumentada}
  
\end{resumo}

\clearpage
\tableofcontents
\thispagestyle{empty}
\clearpage

\pagenumbering{arabic}

\chapter{Resumo do projeto}\label{chp:resultadosEsparados}

\section{Objetivo}

Pretende-se exibir informações e posicionar modelos tridimensionais em uma região do espaço com \textit{AR}, de forma que facilite o acesso do cirurgião à informação durante a cirurgia; estudar e registrar a resposta dos equipamentos utilizados no quesito de qualidade gráfica e latência de resposta do sistema. Tudo isso, com o objetivo central de aumentar a proximidade do cirurgião com a tecnologia de \textit{AR} como apoio durante os procedimentos cirúrgicos.

\section{Metodologia}

Consiste na listagem de possíveis soluções, técnicas ou ferramentas; o estudo e a discussão sobre elas, em seguida, sua implementação. Paralelamente a isso, a busca bibliográfica é constantemente realizada com o objetivo de esclarecer dúvidas sobre os meios imaginados e discutidos com o orientador e coorientador. Essa busca dá ênfase nos resultados encontrados pelos artigos, o objetivo disto é caracterizar os prós e contras das diversas opções encontradas na listagem de técnicas e soluções. Essa pesquisa de artigos desenvolve um discernimento que é refletido em uma noção de funcionamento dos métodos, impactando muito na escolha da implementação para o projeto de pesquisa.

\section{Histórico do projeto}

Desde o início dos trabalhos no projeto, foram experimentados diversos tipos de contato com a elaboração de \textit{softwares} para o sistema operacional \textit{Android} (figura \ref{fig:sceneform}); testes das ferramentas da documentação dos óculos de realidade aumentada \textit{SEIKO EPSON Moverio BT-350} (figura \ref{fig:latinha} e \ref{fig:papercar}); e a elaboração de aplicativos que ilustram o objetivo do \textit{VCranium} (figura \ref{fig:vcranium_alpha}). Assim, como foi explicado no relatório parcial da pesquisa, pretendíamos prosseguir o desenvolvimento estabelecendo uma arquitetura composta por computador, câmera (\textit{webcam)} e óculos para capturar os dados necessários para a projeção em realidade aumentada, mais detalhes serão descritos no capítulo das realizações.

\begin{figure}[ht]
\centering
    \begin{subfigure}{0.45\textwidth}
        \centering
        \includegraphics[width=.95\textwidth]{figuras/sceneform.png}
        \caption{Primeiro aplicativo AR para \textit{Android}}
        \label{fig:sceneform}
    \end{subfigure}
    \begin{subfigure}{0.45\textwidth}
        \includegraphics[width=.95\linewidth]{figuras/Latinha-errada.png}
        \caption{Tentativa de calibração do \textit{Moverio}}
        \label{fig:latinha}
    \end{subfigure}
    \begin{subfigure}{0.45\textwidth}
        \centering
        \includegraphics[width=.95\textwidth]{figuras/PaperCarAR.png}
        \caption{Aplicativo de exemplo fornecido pela \textit{EPSON}}
        \label{fig:papercar}
    \end{subfigure}
    \begin{subfigure}{0.45\textwidth}
        \includegraphics[width=.95\linewidth]{figuras/VCranium.png}
        \caption{Primeira versão do \textit{VCranium}}
        \label{fig:vcranium_alpha}
    \end{subfigure}
    \caption{Histórico de realizações da pesquisa até a primeira entrega parcial. Fonte: Autor.}
    \label{fig:historico}
\end{figure}




\chapter{Realizações}\label{chp:realizacoes}

\section{Estudos em desenvolvimento Android Studio}

O desenvolvimento de aplicativos para \textit{Android} foi estudado nos primeiros meses da pesquisa. Nessa parte do projeto, a pesquisa teve um aspecto mais técnico, que foi necessário para a programação das futuras aplicações que estariam por vir. A \textit{IDE} (ambiente de desenvolvimento integrado) utilizada foi o \textit{Android Studio}, que foi sugerida pelo curso adquirido da \textit{Udemy} \cite{udemy}.

Após as primeiras semanas de aprendizado \textit{Android}, foi iniciado uma pesquisa sobre o desenvolvimento de aplicativos voltados a realidade aumentada. Porém, antes de irmos diretamente nisso, um interessante teste foi realizado com \textit{OpenCV},  que consistiu em realizar as primeira manipulações das imagens da câmera \cite{opencv}. A experiência foi proveitosa para o aprendizado da integração de ferramentas externas ao projeto padrão da plataforma e que, ademais, será feito muitas vezes até a conclusão da pesquisa.

A primeira biblioteca de realidade aumentada a ser testada foi o \textit{Google Sceneform} \cite{Sceneform}. Muitos problemas foram encontrados na integração dos \textit{plugins} com o \textit{Android Studio}, pois este precisava estar em uma versão antiga específica para funcionar. Após a instalação, foi possível ver a primeira projeção em realidade aumentada em um simulador de \textit{smartphone} no computador (figura \ref{fig:sceneform-sim}). Restava testar o aplicativo para um \textit{smartphone} real, porém o afastamento dos integrantes do laboratório pela pandemia, e a falta de dispositivos compatíveis à minha disposição, causaram um atraso nos testes. Felizmente, foi gerado um arquivo que permite a instalação à distância e o aplicativo funcionou com sucesso nos celulares da equipe (figura \ref{fig:sceneform-real}).

\begin{figure}[ht]
\centering
    \begin{subfigure}{.45\textwidth}
        \centering
        \includegraphics[width=.95\textwidth]{figuras/sceneform.png}
        \caption{Simulação da projeção AR no \textit{Android Studio}}
        \label{fig:sceneform-sim}
    \end{subfigure}
    \begin{subfigure}{.45\textwidth}
        \centering
        \includegraphics[width=.95\textwidth]{figuras/sceneformAR.png}
        \caption{\textit{Smartphone} com o aplicativo funcionando}
        \label{fig:sceneform-real}
    \end{subfigure}
    \caption{Resultados adquiridos com o \textit{Google Sceneform}. Fonte: Autor.}
    \label{fig:sceneform-tests}
\end{figure}

Prosseguindo os estudos de aplicativos AR, percebemos que os problemas de versão tidos com o \textit{Sceneform} estavam sendo reparados pelo \textit{Google}e adaptados em um novo programa: \textit{Google ARCore Services} \cite{arcore-googleplay}. Sua proposta é que uma biblioteca seja instalada no celular para que os aplicativos tenham acesso, trazendo a vantagem da redução do tamanho das aplicações produzidas. No entanto, somente uma lista restrita de \textit{smartphones} modernos podem instalar essa biblioteca, a justificativa dos desenvolvedores é a compatibilidade com o sistema \cite{arcore-list}.

Seguido disso, durante a pesquisa foi encontrado uma nova ideia \textit{open-source}, também do \textit{Google}, chamado \textit{Mediapipe}, que tinha a proposta de entregar muitas ferramentas de visão computacional com ML (\textit{machine learning}) integrado \cite{mediapipe-docs}. O projeto foi iniciado em julho de 2019, e tem ganhado mais popularidade por ser gratuito, multi-plataforma e facilitar muito a aplicação de ML para \textit{Android}, \textit{iOS}, e PC. No momento da descoberta, não era conhecido a compatibilidade do \textit{Mediapipe} com \textit{API} antigas de \textit{Android} e muito menos o seu comportamento em óculos de realidade aumentada, por isso, a ideia foi reservada.

\begin{description}
   \item Um sumário das opções testadas com breves comentários abaixo:
   \item[\textit{Google Sceneform}] É um \textit{plugin} para \textit{Android Studio} que introduz muitas ferramentas para o desenvolvimento de aplicativos AR. Funciona somente em \textit{Android} com a \textit{API} 24 ou superior. Seu projeto foi arquivado em meados de 2020. 
   \item[\textit{Google ARCore Services}] Pode se considerar o sucessor das ideias do \textit{Sceneform}. Ele fornece uma biblioteca de ferramentas sofisticadas para AR e atualizações frequentes. Funciona somente em uma lista estrita de \textit{smartphones} modernos. 
   \item[\textit{Mediapipe}] Fornece uma grande quantidade de ferramentas baseadas em \textit{Machine Learning} para visão computacional. Projeto iniciado em junho de 2019 e ganhando mais força recentemente. Pela recente criação, não é conhecido seu comportamento em \textit{API} antigas (28 ou anterior).
\end{description}

% \begin{figure}[ht]
% \centering
%     \begin{subfigure}{0.45\textwidth}
%         \centering
%         \includegraphics[width=.95\textwidth]{figuras/opencv-canny.jpg}
%         \caption{Aplicativo com \textit{OpenCV} integrado no \textit{Android}}
%         \label{fig:canny}
%     \end{subfigure}
%     \begin{subfigure}{0.45\textwidth}
%         \centering
%         \includegraphics[width=.95\textwidth]{figuras/sceneform.png}
%         \caption{Primeiro aplicativo AR para \textit{Android}}
%         \label{fig:sceneform}
%     \end{subfigure}
%     \begin{subfigure}{0.45\textwidth}
%         \centering
%         \includegraphics[width=.95\textwidth]{figuras/PaperCarAR.png}
%         \caption{Aplicativo de exemplo fornecido pela \textit{EPSON}}
%         \label{fig:papercarr}
%     \end{subfigure}
%     \begin{subfigure}{0.45\textwidth}
%         \includegraphics[width=.95\linewidth]{figuras/VCranium.png}
%         \caption{Primeira versão do \textit{VCranium}}
%         \label{fig:vcranium_alpha}
%     \end{subfigure}
%     \caption{Histórico de realizações da pesquisa até a primeira entrega parcial. Fonte: Autor.}
%     \label{fig:historico}
% \end{figure}

O que podemos concluir do desenvolvimento de aplicações em realidade aumentada com \textit{Android Studio} é que mesmo existindo ferramentas robustas de criação de \textit{apps} em AR, eles são restritos às novas versões de \textit{Android}, que por sua vez, estão presentes somente em dispositivos de nova geração, i.e, de lançamento posteriores a 2018. Por fim, a incompatibilidade das bibliotecas com o sistema e a construção dos óculos AR foram os motivos para a procura de um novo ambiente de desenvolvimento, que seja mais versátil e comporte bem com os recursos do \textit{Moverio BT-350}.

\section{Estudos de desenvolvimento Unity}

\textit{Unity} é uma plataforma voltada para o desenvolvimento de jogos em múltiplas plataformas, sendo o \textit{Android} uma delas. Além disso, a USP dá acesso a múltiplos recursos na plataforma para universitários. A documentação da \textit{EPSON} forneceu um programa de exemplo para \textit{Unity} que pode reconhecer um certo carrinho de papel e sobrepor com animações em AR.

Foi sugerido a impressão e a montagem do carrinho de papel pela documentação, porém, como os óculos possuem somente uma \textit{webcam} simples para a detecção, supomos que ele não pode diferenciar o carrinho real de uma fotografia do carrinho, então, ele funcionaria normalmente. De fato, isso ocorreu e o teste foi registrado nas figuras \ref{fig:papercar-stl} e \ref{fig:papercar-ar}.

\begin{figure}[ht]
    \centering
        \begin{subfigure}{.45\textwidth}
            \centering
            \includegraphics[width=.95\textwidth]{figuras/PaperCar.png}
            \caption{Modelo virtual do carrinho de papel}
            \label{fig:papercar-stl}
        \end{subfigure}
        \begin{subfigure}{.45\textwidth}
            \centering
            \includegraphics[width=.95\textwidth]{figuras/PaperCarAR.png}
            \caption{Sobreposição em AR com \textit{Moverio}}
            \label{fig:papercar-ar}
        \end{subfigure}
        \caption{Resultados adquiridos dos testes da documentação \textit{Moverio} np \textit{Unity}. Fonte: Autor.}
        \label{fig:papercar-tests}
\end{figure}

Este foi o único programa disponibilizado na documentação, o projeto foi elaborado no \textit{Unity} 2017, mesmo com alguns alertas de incompatibilidade, o projeto pôde ser compilado com êxito e funcionou nos óculos. Novamente, a falta do suporte da \textit{EPSON} deixou incerto se era uma boa opção construir um aplicativo somente baseado na documentação disponível. Por isso, iniciamos novamente uma pesquisa sobre as ferramentas para suporte de AR nessa plataforma.

O \textit{Vuforia} foi o primeiro \textit{plugin} a ser testado no ambiente do \textit{Unity}, ele tem a opção de integrar uma variedade de novos \textit{targets}, de imagens até cilindros e outras formas 3D. 

\section{Arquitetura do sistema}

Para a definição de uma arquitetura adequada para o projeto foram analisados os exemplos que a literatura pôde nos dar como

[Fazer Tabela de arquiteturas]

Para esse fim, a equipe realizou reuniões e debates para estabelecer uma solução que seja compatível para um período de seis meses e respeitando as medidas de prevenção por afastamento imposto pela pandemia de COVID-19. Decide-se criar sistema que envolve um computador executando um servidor em \textit{Python}; uma 

% que possui ampla compatibilidade com ferramentas atuais de realidade aumentada, e conectar com os óculos executando uma aplicação \textit{Unity} que posicionaria os objetos 3D na visão do usuário sob as instruções do servidor.

Foram estudados diversos tipos de arquitetura do sistema de projeção em realidade aumentada

A escolha da arquitetura do sistema foi baseada na aplicação dos conceitos menos complexos da visão computacional: a detecção da posição de um marcador no espaço. Essa aplicação foi o ponto de partida de estudos das projeções em realidade aumentada em imagens capturadas por uma câmera.

\begin{figure}[ht]
    \centering
    \includegraphics[width=.9\linewidth]{figuras/System schematic.png}
    \caption{A figura representa o funcionamento do sistema. (1) Captura a imagem do paciente e envia para o computador. (2) Faz uma varredura na imagem e identifica o marcador ArUco. (3) Calcula a posição do marcador e envia as coordenadas para os óculos. (4) Recebe as informações e exibe a projeção para o usuário e então retorna para o passo 1. Fonte: Autor.}
    \label{fig:arc}
\end{figure}

% \textbf{Algoritmo de detecção da posição de marcador}
% \textbf{Entrada:} Imagem da câmera
% \textbf{Saída:} Posição do marcador

% \textbf{Etapa 1:} Capture uma imagem da câmera e vá para a Etapa 4, senão há imagem então pare.

% \textbf{Etapa 2:} Identifique o ArUco e vá para a Etapa 3.

% \textbf{Etapa 3:} Estime da posição do marcador no espaço em referência a câmera e retorne à Etapa 1.

% \section{Rascunho}
%% 

% \textit{Vuforia} também tem o suporte do reconhecimento de objetos tridimensionais, porém, os critérios devem ser respeitados:
% \begin{itemize}
%     \item O objeto deve caber no marcador auxiliar (folha A4)
%     \item O objeto não pode conter partes transparentes, translúcidas, luminosas ou reluzentes
%     \item O objeto precisa ter a morfologia (forma) completamente estática
% \end{itemize}

% \begin{figure}[ht]
%     \centering
%     \includegraphics[width=.6\linewidth]{figuras/VCranium.png}
%     \caption{Na esquerda, um teste com um cubo orientado com a face. Na direita, o protótipo do \textit{VCranium} e a exibição do formato de rosto detectado pelo programa. Fonte: Autor.}
%     \label{fig:vcranium}
% \end{figure}

% \begin{figure}[ht]
% \centering
%     \begin{subfigure}{0.45\textwidth}
%         \centering
%         \includegraphics[width=.95\textwidth]{figuras/ConeSTL.png}
%         \caption{Modelo em \textit{STL} do objeto}
%         \label{fig:cone-stl}
%     \end{subfigure}
%     \begin{subfigure}{0.45\textwidth}
%         \includegraphics[width=.95\linewidth]{figuras/Cone.png}
%         \caption{Sessão de captura feita em uma mesa preta}
%         \label{fig:cone-training}
%     \end{subfigure}
%     \begin{subfigure}{0.45\textwidth}
%         \centering
%         \includegraphics[width=.95\textwidth]{figuras/ConeSTL.png}
%         \caption{Modelo em \textit{STL} do objeto}
%         \label{fig:cone-stl}
%     \end{subfigure}
%     \begin{subfigure}{0.45\textwidth}
%         \includegraphics[width=.95\linewidth]{figuras/Cone.png}
%         \caption{Sessão de captura feita em uma mesa preta}
%         \label{fig:cone-training}
%     \end{subfigure}
%     \caption{A tentativa com o cone foi feita em outro ambiente com o fim dar um maior destaque no objeto. Fonte: Autor.}
%     \label{fig:ConeAR}
% \end{figure}

% \begin{center}
% \begin{tabular}{ |c|c| } 
% \hline
% \textbf{Arquivo} & \textit{\textbf{Descrição}} \\ 
% \hline
% \textit{CalibrationTool.apk} & Aplicativo de calibração da projeção da tela dos óculos\\ 
% \hline
% \multirow{2}{*}{\textit{CaptureTool.apk}} & Ferramenta que ajuda a capturar fotos para \\  & o mapeamento de novos marcadores e objetos \\
% \hline
% \multirow{2}{*}{\textit{TrainingToolWindows}} & Usa as imagens obtidas do \textit{CaptureTool.apk} e cria o \\ & \textit{dataset} para a detecção do novo marcador ou objeto  \\
% \hline
% \textit{Moverio\_AR} & Exemplo de cena que aplica funções do \textit{Moverio} no \textit{Unity} \\ 
% \hline
% \end{tabular}
% \end{center}




\chapter{Plano de atividades para o próximo período}\label{chp:plano}

As últimas realizações foram bem acompanhadas por pesquisas novas referências e aprofundamento matemático do problema de calibração encontrado. Ademais, o assunto de calibração é um tema estudado próximo ao anos 2000 e não é discutido atualmente em forma de artigos, mas, em contrapartida, foram encontrados referências na documentação do \textit{OpenCV} e vídeo-aulas como a de Pavel, desenvolvidas na Seção \ref{chp:biblio}. Como um reflexo desse estudo, e a continuação da implementação apresentada na Seção \ref{chp:impl}, definimos um objetivo sólido de prosseguir com a programação da calibração e finalizar as duas etapas (extrínseca e intrínseca) da calibração da projeção em realidade aumentada.

\begin{figure}[H]
   \centering
   \includegraphics[width=.7\linewidth]{figuras/cutoloSys.png}
   \caption{Arquitetura de sistema recentemente publicado. Utiliza conexão sem fio e um \textit{Microsoft Hololens 1} para funcionamento em \textit{OST} e \textit{VST}. Fonte: \cite{Hu2022}}
   \label{fig:cutolo}
\end{figure}

Durante a revisão bibliográfica, foi encontrado um artigo que desenvolve um sistema que auxilia cirurgias ortopédicas utilizando AR. Neste, foram apresentados dois tipos de projeção em realidade aumentada: \textit{VST} ou \textit{video see-through} consiste em exibir um vídeo com a projeção; e \textit{OST} ou \textit{optical see-through} funciona projetando a imagem da sobreposição em um \textit{display} semi-transparente \cite{Hu2022}. 

Ambos os tipos de projeção podem ser feitos com o \textit{Moverio BT-350}, porém, a quantidade de materiais de estudo disponíveis para o \textit{VST} são muito maiores pois é o tipo de projeção mais popular e presente em \textit{smartphones}, \textit{tablets} e óculos de realidade virtual. Por conta disso, foi decidido implementar primeiramente o sistema em \textit{VST}, mesmo os nossos equipamentos permitindo o \textit{OST}. Dessa forma, temos uma maior versatilidade utilizar o \textit{VCranium} também em outros dispositivos além de óculos de realidade aumentada.

\begin{figure}[H]
   \centering
   \includegraphics[width=.65\linewidth]{figuras/phantom.png}
   \caption{A figura demonstra a construção a utilização do calibrador (\textit{phantom}) para a medição do erro da projeção em realidade aumentada. Fonte: \cite{Maruyama2018}}
   \label{fig:phantom}
\end{figure}

Após finalizarmos a calibração da projeção, temos que utilizar um método para medir a sua precisão. Esse campo já foi estudado no artigo de Maruyama, onde foi necessário construir um modelo calibrador que serve como alvo para o algoritmo calcular o erro da visualização, observável na Figura \ref{fig:phantom}. Portanto, a tarefa final do projeto é escolher um método de verificação de erro após a pesquisa e a listagem dos métodos possíveis.

\chapter{Participação de eventos acadêmicos}\label{chp:eventos}

Novamente, os docentes e discentes do laboratório Aerotech se orgulham de todos terem a oportunidade promover nossas pesquisas no 30º Simpósio Internacional de Iniciação Científica e Tecnológica da USP (SIICUSP). Dentre os integrantes, somente um deles foi convidado para a segunda fase que, felizmente, também foi condecorado com uma menção honrosa.

Durante a apresentação, o projeto recebeu algumas provocações que ampliaram as possibilidades de aperfeiçoamento em aspectos de otimização e aplicabilidade. Durante uma apresentação \textit{online} foram respondidas as perguntas:

\begin{description}
    \item[É possível mudar de \textit{Python} para \textit{C/C++} para otimizar processo?] Resposta: É possível, porém isso traz problemas no quesito da velocidade do desenvolvimento, disponibilidade de documentação, bibliotecas e ferramentas. No entanto, é uma ótima opção quando o projeto ter uma versão funcional para testes, tendo um potencial reduzir a latência de cada iteração.
    \item[O quão longe o projeto está de um teste prático?] Resposta: No momento, o projeto consegue fazer demonstrações básicas com as funções de detectar e sobrepor um modelo virtual de um cérebro um marcador em tempo real (aproximadamente 15 vezes por segundo). Porém, o projeto abre muitas possibilidades de ser aperfeiçoado e, por isso, é desejável um trabalho sobre o programa atual antes da apresentação final aos médicos.
\end{description}

As provocações dos avaliadores trouxeram a mensagem que o projeto ainda possui espaço para desenvolver seu algoritmo na questão da otimização e robustez que foram refletidos na Seção \ref{chp:extensao}. Espera-se que a extensão do projeto tenha um perfil voltado a construção de ferramentas para testes práticos do sistema.

% Referências bibliográficas
\printbibliography[heading=bibintoc, title={Referências bibliográficas} ]

\end{document}