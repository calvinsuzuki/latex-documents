\documentclass[Portugues]{projetoFAPESP}

\addbibresource{bibliografia.bib}

\setlength{\parindent}{4em}
\setlength{\parskip}{0.77em}

\usepackage[utf8]{inputenc}
\usepackage[english]{babel}
\usepackage{hyperref}
\usepackage{bookmark}
\usepackage{mathtools}
\usepackage{listings}
\usepackage{multirow}
\usepackage{graphicx}
\usepackage{float}
\usepackage{comment}
\usepackage{subcaption}
\usepackage{indentfirst}
\usepackage{caption}

\usepackage{colortbl}

%\usepackage[table,xcdraw]{xcolor}

% fonte Arial
% \usepackage{helvet}
% \renewcommand{\familydefault}{\sfdefault}

%% Página de título
\universidade{University of São Paulo}
\faculdade{São Carlos School of Engineering}
\cidade{São Carlos}

\titulo{Positioning and display of 3D images using augmented reality glasses for surgical application}

\Proj{2020/15835-4}
\agFomento{UNIVERSITY NAME HERE}{SIGLA}

\professor{Glauco Augusto de Paula Caurin}{Dr.}
\Coorientador{PROFESSOR NAME HERE}{}
\beneficiario{Calvin Suzuki de Camargo}

\inicioPeriodoVigencia{01}{03}{2021}
\periodoRelatorio{28}{02}{2022}
\fimPeriodoVigencia{28}{02}{2022}

\begin{document}

\pagenumbering{roman}

\geraTitulo

\folhaDeRosto

\begin{resumo}
  This is an internship proposal that aims to study the application of smart glasses with augmented reality, which can serve as an auxiliary device in surgical applications. For this, the project deepened concepts of computer vision and computer graphics with the integration between Unity and OpenCV. A system architecture was defined that allows the superimposition of a 3D model on a patient's head, indicating electrode implantation points during the neurosurgical procedure guided by stereoelectroencephalography (SEEG). This architecture was based on communication between the computer and the glasses over the local network: The computer receives the images from the camera on the glasses and uses the detection method using ArUco fiducial markers to estimate the patient's position and sends the coordinates to the glasses, indicating where the projection must be displayed. The partial results of the application show a fast responsiveness with the chosen estimation method.

  \palavraschaves{Smart glasses, Augmented reality, Surgical applications, Computer vision}
\end{resumo}

\clearpage
\tableofcontents
\thispagestyle{empty}
\clearpage

\pagenumbering{arabic}

\chapter{Resumo do projeto}\label{chp:resultadosEsparados}

\section{Objetivo}

Pretende-se exibir informações e posicionar modelos tridimensionais em uma região do espaço com \textit{AR}, de forma que facilite o acesso do cirurgião à informação durante a cirurgia; estudar e registrar a resposta dos equipamentos utilizados no quesito de qualidade gráfica e latência de resposta do sistema. Tudo isso, com o objetivo central de aumentar a proximidade do cirurgião com a tecnologia de \textit{AR} como apoio durante os procedimentos cirúrgicos.

\section{Metodologia}

Consiste na listagem de possíveis soluções, técnicas ou ferramentas; o estudo e a discussão sobre elas, em seguida, sua implementação. Paralelamente a isso, a busca bibliográfica é constantemente realizada com o objetivo de esclarecer dúvidas sobre os meios imaginados e discutidos com o orientador e coorientador. Essa busca dá ênfase nos resultados encontrados pelos artigos, o objetivo disto é caracterizar os prós e contras das diversas opções encontradas na listagem de técnicas e soluções. Essa pesquisa de artigos desenvolve um discernimento que é refletido em uma noção de funcionamento dos métodos, impactando muito na escolha da implementação para o projeto de pesquisa.

\section{Histórico do projeto}

Desde o início dos trabalhos no projeto, foram experimentados diversos tipos de contato com a elaboração de \textit{softwares} para o sistema operacional \textit{Android} (figura \ref{fig:sceneform}); testes das ferramentas da documentação dos óculos de realidade aumentada \textit{SEIKO EPSON Moverio BT-350} (figura \ref{fig:latinha} e \ref{fig:papercar}); e a elaboração de aplicativos que ilustram o objetivo do \textit{VCranium} (figura \ref{fig:vcranium_alpha}). Assim, como foi explicado no relatório parcial da pesquisa, pretendíamos prosseguir o desenvolvimento estabelecendo uma arquitetura composta por computador, câmera (\textit{webcam)} e óculos para capturar os dados necessários para a projeção em realidade aumentada, mais detalhes serão descritos no capítulo das realizações.

\begin{figure}[ht]
\centering
    \begin{subfigure}{0.45\textwidth}
        \centering
        \includegraphics[width=.95\textwidth]{figuras/sceneform.png}
        \caption{Primeiro aplicativo AR para \textit{Android}}
        \label{fig:sceneform}
    \end{subfigure}
    \begin{subfigure}{0.45\textwidth}
        \includegraphics[width=.95\linewidth]{figuras/Latinha-errada.png}
        \caption{Tentativa de calibração do \textit{Moverio}}
        \label{fig:latinha}
    \end{subfigure}
    \begin{subfigure}{0.45\textwidth}
        \centering
        \includegraphics[width=.95\textwidth]{figuras/PaperCarAR.png}
        \caption{Aplicativo de exemplo fornecido pela \textit{EPSON}}
        \label{fig:papercar}
    \end{subfigure}
    \begin{subfigure}{0.45\textwidth}
        \includegraphics[width=.95\linewidth]{figuras/VCranium.png}
        \caption{Primeira versão do \textit{VCranium}}
        \label{fig:vcranium_alpha}
    \end{subfigure}
    \caption{Histórico de realizações da pesquisa até a primeira entrega parcial. Fonte: Autor.}
    \label{fig:historico}
\end{figure}




\chapter{Realizações}\label{chp:realizacoes}

\section{Estudos em desenvolvimento Android Studio}

O desenvolvimento de aplicativos para \textit{Android} foi estudado nos primeiros meses da pesquisa. Nessa parte do projeto, a pesquisa teve um aspecto mais técnico, que foi necessário para a programação das futuras aplicações que estariam por vir. A \textit{IDE} (ambiente de desenvolvimento integrado) utilizada foi o \textit{Android Studio}, que foi sugerida pelo curso adquirido da \textit{Udemy} \cite{udemy}.

Após as primeiras semanas de aprendizado \textit{Android}, foi iniciado uma pesquisa sobre o desenvolvimento de aplicativos voltados a realidade aumentada. Porém, antes de irmos diretamente nisso, um interessante teste foi realizado com \textit{OpenCV},  que consistiu em realizar as primeira manipulações das imagens da câmera \cite{opencv}. A experiência foi proveitosa para o aprendizado da integração de ferramentas externas ao projeto padrão da plataforma e que, ademais, será feito muitas vezes até a conclusão da pesquisa.

A primeira biblioteca de realidade aumentada a ser testada foi o \textit{Google Sceneform} \cite{Sceneform}. Muitos problemas foram encontrados na integração dos \textit{plugins} com o \textit{Android Studio}, pois este precisava estar em uma versão antiga específica para funcionar. Após a instalação, foi possível ver a primeira projeção em realidade aumentada em um simulador de \textit{smartphone} no computador (figura \ref{fig:sceneform-sim}). Restava testar o aplicativo para um \textit{smartphone} real, porém o afastamento dos integrantes do laboratório pela pandemia, e a falta de dispositivos compatíveis à minha disposição, causaram um atraso nos testes. Felizmente, foi gerado um arquivo que permite a instalação à distância e o aplicativo funcionou com sucesso nos celulares da equipe (figura \ref{fig:sceneform-real}).

\begin{figure}[ht]
\centering
    \begin{subfigure}{.45\textwidth}
        \centering
        \includegraphics[width=.95\textwidth]{figuras/sceneform.png}
        \caption{Simulação da projeção AR no \textit{Android Studio}}
        \label{fig:sceneform-sim}
    \end{subfigure}
    \begin{subfigure}{.45\textwidth}
        \centering
        \includegraphics[width=.95\textwidth]{figuras/sceneformAR.png}
        \caption{\textit{Smartphone} com o aplicativo funcionando}
        \label{fig:sceneform-real}
    \end{subfigure}
    \caption{Resultados adquiridos com o \textit{Google Sceneform}. Fonte: Autor.}
    \label{fig:sceneform-tests}
\end{figure}

Prosseguindo os estudos de aplicativos AR, percebemos que os problemas de versão tidos com o \textit{Sceneform} estavam sendo reparados pelo \textit{Google}e adaptados em um novo programa: \textit{Google ARCore Services} \cite{arcore-googleplay}. Sua proposta é que uma biblioteca seja instalada no celular para que os aplicativos tenham acesso, trazendo a vantagem da redução do tamanho das aplicações produzidas. No entanto, somente uma lista restrita de \textit{smartphones} modernos podem instalar essa biblioteca, a justificativa dos desenvolvedores é a compatibilidade com o sistema \cite{arcore-list}.

Seguido disso, durante a pesquisa foi encontrado uma nova ideia \textit{open-source}, também do \textit{Google}, chamado \textit{Mediapipe}, que tinha a proposta de entregar muitas ferramentas de visão computacional com ML (\textit{machine learning}) integrado \cite{mediapipe-docs}. O projeto foi iniciado em julho de 2019, e tem ganhado mais popularidade por ser gratuito, multi-plataforma e facilitar muito a aplicação de ML para \textit{Android}, \textit{iOS}, e PC. No momento da descoberta, não era conhecido a compatibilidade do \textit{Mediapipe} com \textit{API} antigas de \textit{Android} e muito menos o seu comportamento em óculos de realidade aumentada, por isso, a ideia foi reservada.

\begin{description}
   \item Um sumário das opções testadas com breves comentários abaixo:
   \item[\textit{Google Sceneform}] É um \textit{plugin} para \textit{Android Studio} que introduz muitas ferramentas para o desenvolvimento de aplicativos AR. Funciona somente em \textit{Android} com a \textit{API} 24 ou superior. Seu projeto foi arquivado em meados de 2020. 
   \item[\textit{Google ARCore Services}] Pode se considerar o sucessor das ideias do \textit{Sceneform}. Ele fornece uma biblioteca de ferramentas sofisticadas para AR e atualizações frequentes. Funciona somente em uma lista estrita de \textit{smartphones} modernos. 
   \item[\textit{Mediapipe}] Fornece uma grande quantidade de ferramentas baseadas em \textit{Machine Learning} para visão computacional. Projeto iniciado em junho de 2019 e ganhando mais força recentemente. Pela recente criação, não é conhecido seu comportamento em \textit{API} antigas (28 ou anterior).
\end{description}

% \begin{figure}[ht]
% \centering
%     \begin{subfigure}{0.45\textwidth}
%         \centering
%         \includegraphics[width=.95\textwidth]{figuras/opencv-canny.jpg}
%         \caption{Aplicativo com \textit{OpenCV} integrado no \textit{Android}}
%         \label{fig:canny}
%     \end{subfigure}
%     \begin{subfigure}{0.45\textwidth}
%         \centering
%         \includegraphics[width=.95\textwidth]{figuras/sceneform.png}
%         \caption{Primeiro aplicativo AR para \textit{Android}}
%         \label{fig:sceneform}
%     \end{subfigure}
%     \begin{subfigure}{0.45\textwidth}
%         \centering
%         \includegraphics[width=.95\textwidth]{figuras/PaperCarAR.png}
%         \caption{Aplicativo de exemplo fornecido pela \textit{EPSON}}
%         \label{fig:papercarr}
%     \end{subfigure}
%     \begin{subfigure}{0.45\textwidth}
%         \includegraphics[width=.95\linewidth]{figuras/VCranium.png}
%         \caption{Primeira versão do \textit{VCranium}}
%         \label{fig:vcranium_alpha}
%     \end{subfigure}
%     \caption{Histórico de realizações da pesquisa até a primeira entrega parcial. Fonte: Autor.}
%     \label{fig:historico}
% \end{figure}

O que podemos concluir do desenvolvimento de aplicações em realidade aumentada com \textit{Android Studio} é que mesmo existindo ferramentas robustas de criação de \textit{apps} em AR, eles são restritos às novas versões de \textit{Android}, que por sua vez, estão presentes somente em dispositivos de nova geração, i.e, de lançamento posteriores a 2018. Por fim, a incompatibilidade das bibliotecas com o sistema e a construção dos óculos AR foram os motivos para a procura de um novo ambiente de desenvolvimento, que seja mais versátil e comporte bem com os recursos do \textit{Moverio BT-350}.

\section{Estudos de desenvolvimento Unity}

\textit{Unity} é uma plataforma voltada para o desenvolvimento de jogos em múltiplas plataformas, sendo o \textit{Android} uma delas. Além disso, a USP dá acesso a múltiplos recursos na plataforma para universitários. A documentação da \textit{EPSON} forneceu um programa de exemplo para \textit{Unity} que pode reconhecer um certo carrinho de papel e sobrepor com animações em AR.

Foi sugerido a impressão e a montagem do carrinho de papel pela documentação, porém, como os óculos possuem somente uma \textit{webcam} simples para a detecção, supomos que ele não pode diferenciar o carrinho real de uma fotografia do carrinho, então, ele funcionaria normalmente. De fato, isso ocorreu e o teste foi registrado nas figuras \ref{fig:papercar-stl} e \ref{fig:papercar-ar}.

\begin{figure}[ht]
    \centering
        \begin{subfigure}{.45\textwidth}
            \centering
            \includegraphics[width=.95\textwidth]{figuras/PaperCar.png}
            \caption{Modelo virtual do carrinho de papel}
            \label{fig:papercar-stl}
        \end{subfigure}
        \begin{subfigure}{.45\textwidth}
            \centering
            \includegraphics[width=.95\textwidth]{figuras/PaperCarAR.png}
            \caption{Sobreposição em AR com \textit{Moverio}}
            \label{fig:papercar-ar}
        \end{subfigure}
        \caption{Resultados adquiridos dos testes da documentação \textit{Moverio} np \textit{Unity}. Fonte: Autor.}
        \label{fig:papercar-tests}
\end{figure}

Este foi o único programa disponibilizado na documentação, o projeto foi elaborado no \textit{Unity} 2017, mesmo com alguns alertas de incompatibilidade, o projeto pôde ser compilado com êxito e funcionou nos óculos. Novamente, a falta do suporte da \textit{EPSON} deixou incerto se era uma boa opção construir um aplicativo somente baseado na documentação disponível. Por isso, iniciamos novamente uma pesquisa sobre as ferramentas para suporte de AR nessa plataforma.

O \textit{Vuforia} foi o primeiro \textit{plugin} a ser testado no ambiente do \textit{Unity}, ele tem a opção de integrar uma variedade de novos \textit{targets}, de imagens até cilindros e outras formas 3D. 

\section{Arquitetura do sistema}

Para a definição de uma arquitetura adequada para o projeto foram analisados os exemplos que a literatura pôde nos dar como

[Fazer Tabela de arquiteturas]

Para esse fim, a equipe realizou reuniões e debates para estabelecer uma solução que seja compatível para um período de seis meses e respeitando as medidas de prevenção por afastamento imposto pela pandemia de COVID-19. Decide-se criar sistema que envolve um computador executando um servidor em \textit{Python}; uma 

% que possui ampla compatibilidade com ferramentas atuais de realidade aumentada, e conectar com os óculos executando uma aplicação \textit{Unity} que posicionaria os objetos 3D na visão do usuário sob as instruções do servidor.

Foram estudados diversos tipos de arquitetura do sistema de projeção em realidade aumentada

A escolha da arquitetura do sistema foi baseada na aplicação dos conceitos menos complexos da visão computacional: a detecção da posição de um marcador no espaço. Essa aplicação foi o ponto de partida de estudos das projeções em realidade aumentada em imagens capturadas por uma câmera.

\begin{figure}[ht]
    \centering
    \includegraphics[width=.9\linewidth]{figuras/System schematic.png}
    \caption{A figura representa o funcionamento do sistema. (1) Captura a imagem do paciente e envia para o computador. (2) Faz uma varredura na imagem e identifica o marcador ArUco. (3) Calcula a posição do marcador e envia as coordenadas para os óculos. (4) Recebe as informações e exibe a projeção para o usuário e então retorna para o passo 1. Fonte: Autor.}
    \label{fig:arc}
\end{figure}

% \textbf{Algoritmo de detecção da posição de marcador}
% \textbf{Entrada:} Imagem da câmera
% \textbf{Saída:} Posição do marcador

% \textbf{Etapa 1:} Capture uma imagem da câmera e vá para a Etapa 4, senão há imagem então pare.

% \textbf{Etapa 2:} Identifique o ArUco e vá para a Etapa 3.

% \textbf{Etapa 3:} Estime da posição do marcador no espaço em referência a câmera e retorne à Etapa 1.

% \section{Rascunho}
%% 

% \textit{Vuforia} também tem o suporte do reconhecimento de objetos tridimensionais, porém, os critérios devem ser respeitados:
% \begin{itemize}
%     \item O objeto deve caber no marcador auxiliar (folha A4)
%     \item O objeto não pode conter partes transparentes, translúcidas, luminosas ou reluzentes
%     \item O objeto precisa ter a morfologia (forma) completamente estática
% \end{itemize}

% \begin{figure}[ht]
%     \centering
%     \includegraphics[width=.6\linewidth]{figuras/VCranium.png}
%     \caption{Na esquerda, um teste com um cubo orientado com a face. Na direita, o protótipo do \textit{VCranium} e a exibição do formato de rosto detectado pelo programa. Fonte: Autor.}
%     \label{fig:vcranium}
% \end{figure}

% \begin{figure}[ht]
% \centering
%     \begin{subfigure}{0.45\textwidth}
%         \centering
%         \includegraphics[width=.95\textwidth]{figuras/ConeSTL.png}
%         \caption{Modelo em \textit{STL} do objeto}
%         \label{fig:cone-stl}
%     \end{subfigure}
%     \begin{subfigure}{0.45\textwidth}
%         \includegraphics[width=.95\linewidth]{figuras/Cone.png}
%         \caption{Sessão de captura feita em uma mesa preta}
%         \label{fig:cone-training}
%     \end{subfigure}
%     \begin{subfigure}{0.45\textwidth}
%         \centering
%         \includegraphics[width=.95\textwidth]{figuras/ConeSTL.png}
%         \caption{Modelo em \textit{STL} do objeto}
%         \label{fig:cone-stl}
%     \end{subfigure}
%     \begin{subfigure}{0.45\textwidth}
%         \includegraphics[width=.95\linewidth]{figuras/Cone.png}
%         \caption{Sessão de captura feita em uma mesa preta}
%         \label{fig:cone-training}
%     \end{subfigure}
%     \caption{A tentativa com o cone foi feita em outro ambiente com o fim dar um maior destaque no objeto. Fonte: Autor.}
%     \label{fig:ConeAR}
% \end{figure}

% \begin{center}
% \begin{tabular}{ |c|c| } 
% \hline
% \textbf{Arquivo} & \textit{\textbf{Descrição}} \\ 
% \hline
% \textit{CalibrationTool.apk} & Aplicativo de calibração da projeção da tela dos óculos\\ 
% \hline
% \multirow{2}{*}{\textit{CaptureTool.apk}} & Ferramenta que ajuda a capturar fotos para \\  & o mapeamento de novos marcadores e objetos \\
% \hline
% \multirow{2}{*}{\textit{TrainingToolWindows}} & Usa as imagens obtidas do \textit{CaptureTool.apk} e cria o \\ & \textit{dataset} para a detecção do novo marcador ou objeto  \\
% \hline
% \textit{Moverio\_AR} & Exemplo de cena que aplica funções do \textit{Moverio} no \textit{Unity} \\ 
% \hline
% \end{tabular}
% \end{center}




\chapter{Proposta de extensão}\label{chp:extensao}

O projeto até o momento enfrentou diversos desafios e buscou desempenhar muitos testes até decidirmos escolher um meio de começar a elaboração do \textit{VCranium} (Capítulo "\nameref{chp:criacao-vcranium}"). A pesquisa bibliográfica mostrou a possibilidade, por meio de diferentes técnicas, de aplicarmos o \textit{Moverio BT-350} no ambiente cirúrgico e, especialmente, na neurocirurgia \cite{Cho2020}. A extensão do período de trabalho no projeto, assim como a renovação da bolsa, trará também o aprofundamento e refinamento dos processos implementados no programa.

Com a arquitetura do sistema montada, temos um cenário propício para continuar a desenvolver atualizações que permitam uma melhor estabilidade e compatibilidade com os recursos dos óculos. Isto posto, podemos não só refinar o algoritmo atual de detecção por marcadores, como podemos aplicar redes treinadas de \textit{machine learning} ou utilizar o modelo da \textit{Intel RealSense} do laboratório para o uso de \textit{surface matching} da superfície da cabeça do paciente, ambas técnicas já sendo estudadas por outros pesquisadores do \textit{AeroTech}.

\begin{figure}[ht]
    \centering
    \includegraphics[width=.6\linewidth]{figuras/vcranium_calibration.png}
    \caption{Erro de projeção por falta de calibração da câmera. Fonte: Autor.}
    \label{fig:vcranium-calibration}
\end{figure}

O próximo passo do \textit{VCranium} é a calibração da projeção em realidade aumentada. Visto que a observação do usuário dos óculos é estereoscópica, a primeira parte de viabilizar a exibição de AR é fazer a visualização independente para cada olho. Para isso, é necessário um cálculo de calibração particular de cada usuário, e além disso, garantir que essa calibração se mantenha em todas as posições da projeção no espaço virtual. Atualmente, o projeto se encontra desregulado na projeção, não criando o efeito de sobreposição adequado (Figura \ref{fig:vcranium-calibration}).

\section{Objetivo}

Trabalhar na arquitetura de sistema montado no primeiro período da pesquisa, que envolverá os compromissos de refinar o método atual de estimação de posição por marcadores; experimentar diferentes métodos de visão computacional; e comparar com os dados de precisão da literatura. Dessa maneira, cooperando com o objetivo primordial de aumentar a proximidade do neurocirurgião com a tecnologia \textit{AR} em procedimentos cirúrgicos.  

\section{Metodologia}

Como trata-se de um projeto com um sistema já definido, a metodologia precisa dar ênfase no conhecimento da confiabilidade dos métodos e, por isso, consistirá no estudo da matemática das projeções da computação gráfica. Esse assunto é muito importante para entendermos como funciona a visualização 3D de objetos em um espaço virtual, os cálculos são feitos com transformações por meio da multiplicação de matrizes (\textit{Model, View} e \textit{Projection}), que por sua vez, agem como parâmetros desse método \cite{learnopengl-coord-sys}.

\begin{figure}[ht]
   \centering
   \includegraphics[width=.65\linewidth]{figuras/coordinate_systems.png}
   \caption{Aplicação das matrizes \textit{Model, View} e \textit{Projection} para a visualização 3D em perspectiva. Fonte: \cite{learnopengl-coord-sys}.}
   \label{fig:coord-sys}
\end{figure}

Não se limitando apenas na teoria, mas também estudando a aplicação da computação gráfica em realidade aumentada, podemos então criar uma relação entre as visualizações 3D do espaço virtual dos óculos e a tela em que serão projetadas as imagens, sendo essa, a base do efeito de sobreposição em \textit{AR}. Portanto, esse trabalho vai enfatizar os estudos de marcadores fiduciais para essa visualização, sendo necessário correlacionar os parâmetros da câmera com os modelos com o intuito de fazer uma projeção de forma adequada \cite{learningOpenGL}.

Por fim, o trabalho exigirá uma aferição da precisão da projeção em realidade aumentada apresentada pelo método. Na literatura, essa medida foi adquirida com a utilização de ferramentas que entregam uma estimação do erro da projeção (Figura \ref{fig:phantom}) \cite{Maruyama2018}. Ainda que a metodologia terá destaque no aprofundamento teórico dos métodos, as pesquisas de novos artigos serão feitas conjuntamente com esses estudos, procurando monitorar as descobertas e novos resultados no campo de realidade aumentada aplicada em cirurgias e neurocirurgias.

\begin{figure}[ht]
   \centering
   \includegraphics[width=.5\linewidth]{figuras/phatom.png}
   \caption{Aplicação do \textit{phantom} para a medição do erro de projeção. Fonte: \cite{Maruyama2018}.}
   \label{fig:phantom}
\end{figure}



\newpage

\section{Cronograma}

O controle de atividades vai ser feito com encontros mensais e semanais com o coorientador, orientador e integrantes do laboratório. Um cronograma foi elaborado na tabela \ref{fig:tabela} para servir de guia para o progresso do projeto.

\begin{table}[h]
    \begin{tabular}{llllllll} 
        \hline
\multicolumn{8}{c}{\textbf{Cronograma de atividades para a extensão}} \\ 
\hline
& & & {\cellcolor[rgb]{.3,.3,1}} & \multicolumn{4}{l}{Execução} \\ 
    \hline
 \textbf{Atividade} & \textbf{Bimestre} & 1 & 2 & 3 & 4 & 5 & 6 \\ 
    \hline
 \begin{tabular}[c]{@{}>{}l@{}}Revisão literária de aplicação de marcadores em AR \\\end{tabular} & & {\cellcolor[rgb]{.3,.3,1}} & & & & & \\ 
    \hline
 Estudo das transformações matriciais de perspectiva & & {\cellcolor[rgb]{.3,.3,1}} & {\cellcolor[rgb]{.3,.3,1}} & & & & \\ 
    \hline
 Estudo sobre calibração de projeção AR & & & {\cellcolor[rgb]{.3,.3,1}} & {\cellcolor[rgb]{.3,.3,1}} & & & \\ 
    \hline
 Aprofundamento em calibração estereoscópica Moverio & & & & {\cellcolor[rgb]{.3,.3,1}} & & & \\ 
    \hline
 Testes com outros algoritmos de detecção & & & & & {\cellcolor[rgb]{.3,.3,1}} & {\cellcolor[rgb]{.3,.3,1}} & \\ 
    \hline
 Coleta de dados da precisão de métodos* & & & & & {\cellcolor[rgb]{.3,.3,1}} & {\cellcolor[rgb]{.3,.3,1}} & {\cellcolor[rgb]{.3,.3,1}}  \\
    \hline
    \end{tabular}
    \caption{*A coleta de dados vai envolver o auxílio da parceria do Centro de Cirurgia de Epilepsia (CIREP) do Hospital das Clínicas da Faculdade de Medicina de Ribeirão Preto-USP}
    \label{fig:tabela}
\end{table}

\chapter{Participação de eventos acadêmicos}\label{chp:eventos}

Novamente, os docentes e discentes do laboratório Aerotech se orgulham de todos terem a oportunidade promover nossas pesquisas no 30º Simpósio Internacional de Iniciação Científica e Tecnológica da USP (SIICUSP). Dentre os integrantes, somente um deles foi convidado para a segunda fase que, felizmente, também foi condecorado com uma menção honrosa.

Durante a apresentação, o projeto recebeu algumas provocações que ampliaram as possibilidades de aperfeiçoamento em aspectos de otimização e aplicabilidade. Durante uma apresentação \textit{online} foram respondidas as perguntas:

\begin{description}
    \item[É possível mudar de \textit{Python} para \textit{C/C++} para otimizar processo?] Resposta: É possível, porém isso traz problemas no quesito da velocidade do desenvolvimento, disponibilidade de documentação, bibliotecas e ferramentas. No entanto, é uma ótima opção quando o projeto ter uma versão funcional para testes, tendo um potencial reduzir a latência de cada iteração.
    \item[O quão longe o projeto está de um teste prático?] Resposta: No momento, o projeto consegue fazer demonstrações básicas com as funções de detectar e sobrepor um modelo virtual de um cérebro um marcador em tempo real (aproximadamente 15 vezes por segundo). Porém, o projeto abre muitas possibilidades de ser aperfeiçoado e, por isso, é desejável um trabalho sobre o programa atual antes da apresentação final aos médicos.
\end{description}

As provocações dos avaliadores trouxeram a mensagem que o projeto ainda possui espaço para desenvolver seu algoritmo na questão da otimização e robustez que foram refletidos na Seção \ref{chp:extensao}. Espera-se que a extensão do projeto tenha um perfil voltado a construção de ferramentas para testes práticos do sistema.

% Referências bibliográficas
\printbibliography[heading=bibintoc, title={References} ]

\end{document}