\chapter{Participação de eventos acadêmicos}\label{chp:eventos}

Novamente, os docentes e discentes do laboratório Aerotech se orgulham de todos terem a oportunidade promover nossas pesquisas no 30º Simpósio Internacional de Iniciação Científica e Tecnológica da USP (SIICUSP). Dentre os integrantes, somente um deles foi convidado para a segunda fase que, felizmente, também foi condecorado com uma menção honrosa.

Durante a apresentação, o projeto recebeu algumas provocações que ampliaram as possibilidades de aperfeiçoamento em aspectos de otimização e aplicabilidade. Durante uma apresentação \textit{online} foram respondidas as perguntas:

\begin{description}
    \item[É possível mudar de \textit{Python} para \textit{C/C++} para otimizar processo?] Resposta: É possível, porém isso traz problemas no quesito da velocidade do desenvolvimento, disponibilidade de documentação, bibliotecas e ferramentas. No entanto, é uma ótima opção quando o projeto ter uma versão funcional para testes, tendo um potencial reduzir a latência de cada iteração.
    \item[O quão longe o projeto está de um teste prático?] Resposta: No momento, o projeto consegue fazer demonstrações básicas com as funções de detectar e sobrepor um modelo virtual de um cérebro um marcador em tempo real (aproximadamente 15 vezes por segundo). Porém, o projeto abre muitas possibilidades de ser aperfeiçoado e, por isso, é desejável um trabalho sobre o programa atual antes da apresentação final aos médicos.
\end{description}

As provocações dos avaliadores trouxeram a mensagem que o projeto ainda possui espaço para desenvolver seu algoritmo na questão da otimização e robustez que foram refletidos na Seção \ref{chp:extensao}. Espera-se que a extensão do projeto tenha um perfil voltado a construção de ferramentas para testes práticos do sistema.