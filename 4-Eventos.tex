\chapter{Participação de eventos acadêmicos}\label{chp:disseminacao}

Novamente, os docentes e discentes do laboratório Aerotech se orgulham de todos terem a oportunidade promover nossas pesquisas no 30º Simpósio Internacional de Iniciação Científica e Tecnológica da USP (SIICUSP). Dentre os integrantes, somente um deles foi convidado para a segunda fase que, felizmente, também foi condecorado com uma menção honrosa.

Durante a apresentação, o projeto recebeu algumas provocações que ampliaram as possibilidades de aperfeiçoamento em aspectos de otimização e aplicabilidade. Durante uma apresentação \textit{online} foram respondidas as perguntas:

\begin{description}
    \item[É possível mudar de \textit{Python} para \textit{C/C++} para otimizar processo?] Resposta: É possível, porém isso traz problemas no quesito da velocidade do desenvolvimento, disponibilidade de documentação, bibliotecas e ferramentas. No entanto, é uma ótima opção quando o projeto ter uma versão funcional para testes, tendo um potencial reduzir a latência de cada iteração.
    \item[O quão longe o projeto está de um teste prático?] Resposta: No momento, o projeto consegue detectar a posição de marcadores em tempo real e posicionar um modelo virtual do \textit{Unity} (aproximadamente 15 ciclos por segundo). Porém, utilizar somente um marcador traz diversas limitações de movimento ao usuário dos óculos, portanto, é necessário uma adaptação do sistema para a ampliar seu alcance e superar os problemas de oclusão do marcador.
\end{description}