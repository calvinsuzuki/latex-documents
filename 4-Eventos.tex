% \chapter{Participação de eventos acadêmicos}\label{chp:disseminacao}

% O laboratório Aerotech teve que entregar um relatório geral das atividades nos alunos de iniciação científica e esse documento teve uma seção reservada para isso em "Robótica colaborativa e neuronavegação aplicados à neurocirurgia"\, \cite{Caurin2020}. Outra oportunidade de promoção do projeto aconteceu no 28º Simpósio Internacional de Iniciação Científica e Tecnológica da USP (SIICUSP), o projeto teve algumas críticas na primeira etapa (etapa nacional) como a menção de "resultado satisfatório"\, na performance das imagens \ref{fig:vuforia}, sem a devida explicação dos critérios tomados e a falta da análise quantitativa. 

% Mesmo com a crítica, o projeto teve a aprovação para seguir para a segunda fase (fase internacional), tendo um discurso mais limpo após as críticas. Além da indicação para a segunda fase, foi recebido um convite para a participação no \textit{Flash Talks}, que consistia na divulgação científica para o público leigo no assunto, para isso, foi gravado um vídeo apresentando o projeto, porém não foi obtido nenhuma glorificação. No final, o projeto do SIICUSP foi satisfatoriamente aprovado na segunda fase do simpósio, consagrando uma menção honrosa a mim e evidenciando a relevância que o projeto possui. 

% Após essa última aprovação, um outro convite foi dado pela Sociedade Brasileira para o Progresso da Ciência (SBPC) para a participar da 73ª Jornada Nacional de Iniciação Científica (JNIC). Infelizmente, o projeto foi recusado por não respeitar os critérios dados pelo edital por se tratar de um "trabalho sem resultado".
