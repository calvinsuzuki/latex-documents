 \chapter{Introdução}\label{chp:intro}

Usinas hidrelétricas respondem pela maior parte da geração energética do Brasil, de modo que a otimização do rendimento dessas plantas é fundamental para manter a estabilidade da rede, preservando os níveis de água dos reservatórios, e diminuindo o custo de operação.
No presente trabalho, busca-se fazer uma análise multidimensional dos parâmetros de projeto de uma turbina Francis, desde a econometria por trás da implementação nessa escala até a modelagem do funcionamento e dimensionamento de parâmetros gerais da turbina, visando otimizar o rendimento energético mecânico.
Serão avaliados também os impactos de parâmetros críticos na eficiência do sistema, direcionando as escolhas do projeto.