\chapter{\textit{Benchmarks}}\label{chp:benchmk}

\subsection{Plataformas}

Infelizmente não se apresentam na comunidade e na literatura grandes quantidades de \textit{benchmarks} contendo os trabalhos que foram desenvolvidos e os resultados alcançados. Entretanto, existem algumas plataformas que estão crescendo e apresentam a tarefa de detecção de crises epilépticas com alguns trabalhos catalogados. Entre elas o \textit{Papers With Code} \cite{paperswithcode}, ou também o \textit{Kaggle} \cite{kaggle}. 

\subsection{Métricas}

Para avaliar os modelos, a literatura utiliza métricas como a taxa de acertos, os falsos positivos e falsos negativos \cite{godoyprediccao}. Já que a tarefa é uma classificação binária, os meios de avaliação da resposta são mais simples em relação a outras tarefas de aprendizado de máquina. Além disso, métodos de avaliação dos modelos como a complexidade, a curva de treinamento e a avaliação de hiper-parâmetros permitem a análise mais aprofundada do modelo e promove consequências diretas nos resultados.